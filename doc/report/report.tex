\documentclass[11pt, a4paper]{article}

\usepackage{amsmath}
\usepackage{amsopn}
\usepackage{ctex}
\usepackage{geometry}
\usepackage{alltt}
\usepackage{color}
\usepackage{fullpage}
\usepackage{framed}
\usepackage{hyperref}
\usepackage{clrscode3e}

\definecolor{string}{rgb}{0.7,0.0,0.0}
\definecolor{comment}{rgb}{0.13,0.54,0.13}
\definecolor{keyword}{rgb}{0.0,0.0,1.0}

\begin{document}
\title{数值最优化实验报告:多种线搜索算法的实现和比较}
\author{15336134 莫凡}
\maketitle
\vspace{20pt}
\begin{center}
	\Large
	Version II.
\end{center}
\vspace{30pt}

\begin{abstract}
	本文是中山大学2015级信息与计算科学专业的运筹学与最优化课程作业\cite{高立2014数}。内容为使用各种基于线搜索的牛顿方法实现对Waston函数\cite{More:1981:TUO:355934.355936}的优化。本文主要使用MATLAB R2016b实现并对比了非精确线搜索的(Strong) Wolfe准则、Wolfe准则、Goldstein准则与armijo准则,应用了最速下降法、阻尼牛顿法、LM方法、拟牛顿法(SR1公式、DFP公式、BFGS公式)测量了n=6,9,12的情形。
	
	为了保证算法的正确性,本文同时根据\cite{More:1981:TUO:355934.355936}的建议,实现了两个小函数用来检验算法的正确性。通过实验,发现在不使用Armijo准则的条件下,6阻尼牛顿法、LM方法、SR1公式、BFGS公式均能达到预期的收敛效果,而DFP公式和最速下降法没有在指定迭代次数内收敛。阻尼牛顿法和拟牛顿法显然比拟牛顿方法具有更快的收敛速度。对比多种线搜索步长方法,使用强Wolfe准则是相对效果较为出色的方案。
\end{abstract}
\newpage
\tableofcontents
\newpage

\section{问题的分析与重述}

\subsection{环境与假设}
本次试验主要在Windows8.1 Home Basic操作系统下,使用MATLAB R2016b完成。为了对于函数值与梯度进行检验,也同时用到了以下软件或程序包
\begin{itemize}
	\item R 3.3.4
	\item Wolfram Mathematica 10
\end{itemize}

本文所涉及的所有代码与参考文献(如果能搜集到电子版)全部托管在 \url{https://github.com/w007878/watson},以MIT协议发布

本文所完成的任务,是通过基于线搜索的迭代方法,求得一个函数在定义域上的数值极小值。由于函数的复杂性,我们无法验证此极小值是否为全局最小值,只能借助文献\cite{More:1981:TUO:355934.355936}中的参考值来检验结果的正确性。

为了评估一项算法,我们采用一下几个指标
\begin{enumerate}
	\item 算法是否在规定迭代次数内收敛,如果是,需要迭代多少步
	\item 算法停止后(包括满足梯度条件或者迭代次数条件终止)是否得到正确的极小值
	\item 函数调用了多少次
	\item MATLAB的cputime
\end{enumerate}

\subsection{Watson 函数的定义与计算}
考虑以下定义在n维空间上的实值函数\[f(x)=\sum_{i=1}^m r_i^2(x)\]其中$n,\,m,\,r_i$可以是各种各样的取值。现要实现一个优化算法,求解这个函数的极小值。

本次试验的主要任务是实现\[r_i(x)=\sum_{j=2}^n\,(j-1)\,x_j\,t_i^{j-2}-\left(\sum_{j=1}^n\,x_j\,t_i^{j-1}\right)^2-1\]的情况。除此之外,还要有
\[m=31\quad 2\le n\le 31\]
\[t_i=\frac{i}{29}\quad 1\le i\le 29\]
\[r_{30}(x)=x_1\quad r_{31}(x)=x_2-x_1^2-1\]

这个函数被称为Watson函数\cite{dennis1996numerical},是我们要优化的目标。首先我们要得到的就是它的梯度与Hessian矩阵的表达式。事实上,我们没有必要写出$\nabla f$关于自变量$x$的表达式,而是用$r(x)$和它的偏导数进行代替
\[\frac{\partial f}{\partial x_k}=2\sum_{i=1}^m r_i(x)\frac{\partial r_i}{\partial x_k}\]

刨除最后两项不优美的式子,我们有
\[\frac{\partial f}{\partial x_k}=\sum_{i=1}^m\frac{\partial f}{\partial r_i}\frac{\partial r_i}{\partial x_k}\]
其中
\[\frac{\partial r_i}{\partial x_k}=(k-1)\,t_i^{k-2}-2\left(\sum_{j=1}^nx_jt_i^{j-1}\right)\,t_i^{k-1}\]

为了在代码中计算出这个东西,我们构造两个辅助矩阵。(为简单起见,用m代替29)
\begin{equation*}
T_1=
\begin{bmatrix}
t_1^0 & t_1^1 & \cdots & t_1^{n-1}\\
t_2^0 & & \cdots & t_2^{n-1}\\
\cdots & \cdots &\cdots & \cdots\\
t_m^0 & \cdots &\cdots & t_m^{n-1}
\end{bmatrix}
\qquad 
T_2=
\begin{bmatrix}
0 & t_1^0 & t_1^1 & \cdots & t_1^{n-2}\\
0 & t_2^0 & & \cdots & t_2^{n-2}\\
& \cdots & \cdots &\cdots & \cdots\\
0 & t_m^0 & \cdots &\cdots & t_m^{n-2}
\end{bmatrix}
\end{equation*}

得到MATLAB代码

\begin{framed}
\begin{alltt}
	01	tmp1 = 4 * T1' * (r(1:29) .* (T1 * x));
	02	tmp2 = 2 * (0:n-1)' .* (T2' * r(1:29));
	03	grad = tmp2 - tmp1;
\end{alltt}
\end{framed}

然后根据$r_{30}$与$r_{31}$两项,单独处理一下$\dfrac{\partial f}{\partial x_1}$与$\dfrac{\partial f}{\partial x_2}$,梯度就计算完成了

接下来是Hessian矩阵。有表达式
\[
\frac{\partial^2 f}{\partial x_b \partial x_a}=2\sum_{i=1}^{m}\left(\frac{\partial r_i}{\partial x_b}\,\frac{\partial r_i}{\partial x_a}+\frac{\partial^2r_i}{\partial x_b\,\partial x_a}\,r_i(x)\right)
\]

这时我们采用最简单暴力的方法单独计算每一项偏导数以得到矩阵

\begin{framed}
\begin{alltt}
	01 prz = (0:n-1) .* T2 - 2 * (T1 * x) .* T1;
	02 for a = 1:n
	03     for b = 1:n
	04 	       H(a, b)=2*prz(:,a)'*prz(:,b)+2*(-2*power(t,a+b-2))*r(1:29);
	05     end
	06 end
\end{alltt}
\end{framed}

代码中的prz表示的是每个r对每个x的偏导数。

通过对计算复杂度的分析,我们发现计算函数值、梯度和Hessian矩阵并没有很大的区别。同时,梯度和Hessian矩阵都需要用到相同的中间变量。而由于$n$很小,所以这点复杂度的差别基本忽略不计,于是使用一个函数同时计算出他的函数值、梯度与Hessian矩阵。

\section{测试环节}

\subsection{三个测试函数}

为了验证优化算法的正确性,在使用它优化Watson函数之前使用几个简单的函数进行测试。

这几个函数都具有\(f(x)=\sum_{i=1}^m r_i^2(x)\)的形式。

\subsubsection{Rosenbrock function}
Rosenbrock function\cite{doi:10.1093/comjnl/3.3.175}满足
\begin{equation*}
\begin{array}{c}
n=2\quad m=2\\
r_1(x)=10(x_2-x_1^2)\\
r_2(x)=1-x_1
\end{array}
\end{equation*}

若以点$(-1.2, 1)$作为迭代起点,这个函数应当收敛到点$(1, 1),~~f=0$
\subsubsection{Freudenstein-and-Roth function}
Freudenstein-and-Roth function\cite{freudenstein1963numerical}满足
\begin{equation*}
\begin{array}{c}
n=2\quad m=2\\
r_1(x)=-13+x_1+((5-x_2)x_2-2)x_2\\
r_2(x)=-29+x_1+((x_2+1)x_2-14)x_2
\end{array}
\end{equation*}

若以点$(0.5,-2)$作为迭代起点,可能会得到1. 局部极小值$(11.41\cdots,-0.8968\cdots),~~f=48.9842$\quad 2.全局最小值$(5,4),~~f=0$
\subsubsection{Powell-badly-scaled function}
Powell-badly-scaled function\cite{Powell1970}满足
\begin{equation*}
\begin{array}{c}
n=2\quad m=2\\
r_1(x)=10^4x_1x_2-1\\
r_2(x)=e^{-x_1}+e^{-x_2}-1.0001
\end{array}
\end{equation*}
若以点$(0,1)$作为迭代起点,应当得到极小值$(1.908\cdots 10^{-5},9.106\cdots),~~f=0$

\subsection{辅助软件检测} 

使用R语言检车Watson函数的数值,用它的numDeriv包检测MATLAB程序中计算的梯度和Hessian矩阵的数值。

使用Mathematica来检查以上三个测试函数的梯度与Hessian矩阵的解析形式是否正确。

\section{线搜索程序}

假设$d$是线搜索下降方向(Newton/quasi-Newton 方向),我们要求得正实数$\alpha$,使得$f(x_k+\alpha d)\le f(x_k)$,然后更新$x_{k+1}=x_k+\alpha d$

线搜索的程序都比较简单。所以不需要进行额外的测验。实现之后可以直接用于后文中的优化算法。

\subsection{定步长}
基本牛顿方法直接设步长$\alpha=1$。如果采用这种方式,应当将下降方向向量正则化之后再应用。

\subsection{精确线搜索}
令\[\alpha=\mathop{\arg \min}\limits_\alpha f(x_k+\alpha d)\]
这里直接使用MATLAB内置的优化函数

\subsection{非精确线搜索}

我们使用最常规的“回溯法”求得非精确线搜索步长。伪代码如下
\begin{codebox}
\Procname{$\proc{Linear-Search}(d,\rho,\epsilon,\alpha_0)$}
\li $\alpha\gets\alpha_0$
\li \While not $\proc{Is-Satisfy-Condition}(d,\alpha)$
	\Do
\li		$\alpha\gets\rho\alpha$
\li		\If $\alpha<\epsilon$
\li			\Then  \Return $\alpha$		
		\End
	\End
\li \Return $\alpha$
\end{codebox}

其中$\epsilon$是用于限制步长不应小于某个数的阈值,以防算法收敛速度过慢。代码中使用了$10^{-5}$。$\rho$是用于更新$\alpha$的值,在实践中取了$0.99~0.999~0.99999999$,事实证明因为下降的速度是指数,所以没有太大关系


\subsubsection{Armijo Condition}
\[f(x_k+\alpha p_k)\le f(x_k)+c\alpha p_k^T\]

\subsubsection{Wolfe Condition}
\[f(x_k+\alpha p_k)\le f(x_k)+c_1\alpha\nabla f_k^Tp_k\]
\[\nabla f(x_k+\alpha_kp_k)^Tp_k \ge c_2\nabla f_k^Tp_k\]

\subsubsection{Strong Wolfe Condition}
\[f(x_k+\alpha p_k)\le f(x_k)+c_1\alpha\nabla f_k^Tp_k\]
\[|\nabla f(x_k+\alpha p_k)^Tp_k| \le c_2|\nabla f_k^Tp|\]

根据前人的经验\cite{ding2005investigation}\cite{Nocedal2006NO}\cite{gilbert1989some},Armijo准则中的$c$、(Strong)Wolfe准则中的$c_1$取值$10^{-4}$(前二者),$10^{-3}$(后者),完全是没有理论依据的经验,我们这里选择使用$10^{-4}$(Strong) Wolfe准则中的$c_2$取值$0.9$,可以得到更好的步长效果

\subsubsection{Goldstein Condition}
\[f(x_k)+(1-c)\nabla f_k^Tp_k\alpha \le f(x_k+\alpha p_k)\le f(x_k)+c\nabla f_k^T p_k\alpha\]


根据\cite{shanno1970conditioning}与\cite{ZHANG2001269}的建议,使用线搜索的Newton类方法较为常用的是满足(Strong) Wolfe Condition的非精确步长。

在\cite{Nocedal2006NO}中,提到了一种借助两点三次埃尔米特插值求得满足强Wolfe条件的线搜索步长的方法。然而\cite{ding2005investigation}实现了这种方法,在这个问题上和本文中的朴素方法并没有明显的区别,所以在本文的实验中并没有采用这种方法。

关于朴素线搜索的步长,在牛顿拟牛顿类方法一个通常的做法是设置$\alpha_0=1$,但是在\cite{Nocedal2006NO}中提到了一种类似插值的初始长度策略,即设$\phi'(0)=\nabla f_k^Tp_k$,令
\[\alpha_0=\min\left(1,~~1.01\times \frac{2(f_k-f_{k-1})}{\phi'(0)}\right)\]

接下来的工作中会对这两种方法进行进一步的比较。

\section{优化算法}

\subsection{阻尼牛顿法}

牛顿方向是\[d=-G^{-1}g\]

其中$G$是Hessian矩阵,$g$是梯度

阻尼牛顿法是在牛顿方向上,通过先搜索得到合适的步长,然后进行迭代。

迭代停止的条件为i.迭代超过一定次数(不收敛)~~ii.梯度的二范数小于阈值$\epsilon$(收敛)。而对于阻尼牛顿法,当iii.Hessian矩阵$G$奇异,或接近奇异.iv.Hessian矩阵不正定时,应当终止迭代s

此类方法的伪代码是

\begin{codebox}
	\Procname{$\proc{Damped-Newton}(x_0,\proc{func},C_{imax},\epsilon)$}
	\li $x\gets x_0$
	\li $time\gets cputime$
	\li $C_f\gets 1$
	\li $C_i\gets 0$
	\li $[f,g,G]\gets \proc{func}(x)$
	\li \While $g^Tg>\epsilon$ 
		\Do
	\li 	$C_i\gets C_i+1$
	\li 	\If $C_i>C_{imax}$ or $|G|==0$ or $\min (\attrib{G}{eig-values})\le0$
	\li			\Then \Return f
			\End
	\li		$d\gets -G^{-1}g$
	\li 	$\alpha\gets\proc{Linear-Search}(d,\const{1e-5},\const{0.9999},\const{1})$
	\li 	$x\gets x+\alpha d$
	\li 	$[f,g,G]\gets\proc{func}(x)$
		\End
	\li $x_{min}\gets x$
	\li $time\gets cputime-time$
	\li \Return $(f,x_{min},C_i,C_f,time)$
\end{codebox}

\subsection{修正牛顿法}

我们采用LM算法(Levenberg-Marquart Algorithm\cite{more1978levenberg}\cite{ranganathan2004levenberg})来处理Hessian矩阵非正定的情况。方案是在$G$非正定的情况下,将迭代方向修正为\[d=-(G+\lambda I)^{-1}g\]其中$\lambda$是$G$的最小特征值的相反数$+10^{-5}$

此时的结束条件就不需要判断矩阵是否奇异或非正定。

\subsection{拟牛顿方法}

满足拟牛顿条件\[s_k=H_{k+1}y_k\quad s_k=x_{k+1}-x_k\quad y_k=g_{k+1}-g_k\]$H$是一个对称正定矩阵,\cite{高立2014数}给出的初始值是令$H_0$为单位阵。然而,\cite{conn1991convergence}表明,将Hessian矩阵作为初始值会使得拟牛顿方法的收敛速度加快。为了防止初始Hessian矩阵非正定,由类似于LM方法给出它的初值。$H_k$由以下的迭代公式给出

\subsubsection{SR1公式}
\[H_{k+1}=H_k+\frac{(s_k-H_ky_k)(s_k-H_ky_k)^T}{(s_k-H_ky_k)^Ty_k}\]

\subsubsection{DFP公式}
\[H_{k+1}=H_k+\frac{s_ks_k^T}{s_k^Ty_k}-\frac{H_ky_ky_k^TH_k}{y_k^TH_ky_k}\]

\subsubsection{BFGS公式}
\[H_{k+1}=H_k+\left(1+\frac{y_k^TH_ky_k}{y_k^Ts_k}\right)\frac{s_ks_k^T}{y_k^Ts_k}-\left(\frac{s_ky_k^TH_k+H_ky_ks_k^T}{y_k^Ts_k}\right)\]

\section{实验与结果}
\subsection{文件函数清单}

\paragraph{线搜索程序}

\begin{tabular}{|l|r|}
	\hline
	acc.m & 精确线搜索\\
	naive\_armijo.m & armijo准则非精确线搜索\\
	naive\_goldstein.m & goldstein准则非精确线搜索\\
	naive\_wolfe.m & wolfe准则非精确线搜索\\
	naive\_strong\_wolfe.m & strong wolfe准则非精确线搜索\\\hline
\end{tabular}

\paragraph{函数程序}

\begin{tabular}{|l|r|}
	\hline
	freudenstein\_roth.m & 检验函数\\
	Rosenbrock.m & 检验函数\\
	powell.m & 检验函数\\
	watson.m & 待优化函数\\\hline
\end{tabular}

\paragraph{优化程序}

\begin{tabular}{|l|r|}
	\hline
	gradient\_descent.m & 梯度下降\\
	damped\_newton.m & 阻尼牛顿法\\
	lm.m & LM算法\\
	sr1.m & SR1公式\\
	dfp.m & DFP公式\\
	bfgs.m & BFGS 公式\\\hline
\end{tabular}

\subsection{用检验函数检验}

对于除DFP以外地所有的方法,对于Rosenbrock函数收敛到期望值,但是需要设置足够高的精度才能使得收敛效果较好,而阻尼牛顿法和BFGS公式是收敛最快的。

SR1、DFP、BFGS方法在freudenstein\_roth函数收敛到全局最小值,阻尼牛顿方法指定步数内收敛到局部极小值,LM指定步数内不收敛

在Powell函数上所有方法出现了问题(正定性)。\cite{Byrd1988}指明信赖域方法在此函数更为有效。

\subsection{对比优化方法}

接下来的内容中分别对每一种求线搜索步长的方法测试各种优化函数。根据\cite{More:1981:TUO:355934.355936}提供的数据,我们只进行$n=6,\,9,\,12$时的数据实验。因为在$n$取更大的值是,会由于Hessian矩阵接近奇异而无法计算。目前并没有找到有人做过成功的相关实验。并且随着$n$的增长,极小值以很高的精度趋近于0但恒大于0。在$n$极大的情况下,即使能收敛,也难以达到所要求的数值精度

\subsubsection{固定步长与精确线搜索}

效果极差,在指定步数内基本不收敛

\subsubsection{Armijo准则}

n=6,最大迭代次数1000,$\epsilon=10^{-10}$

\vspace{10pt}
\begin{tabular}{|l|c|c|c|c|c|}
	\hline
	方法 & 阻尼 & LM & SR1 & DFP & BFGS \\\hline
	是否准确收敛 & $\surd$ & $\surd$ & $\surd$ & $\times$ & $\surd$ \\\hline
	迭代次数 & 11 & 11 & 79 & - & 45 \\\hline
	函数调用次数 & 35 & 35 & 6339 & - & 3189 \\\hline
	CPU时间 & 0.0156 & 0 & 0.5781 & - & 0.1719\\\hline
\end{tabular}

\vspace{20pt}
n=9,最大迭代次数1000,$\epsilon=10^{-100}$

\vspace{10pt}
\begin{tabular}{|l|c|c|c|c|c|}
	\hline
	方法 & 阻尼 & LM & SR1 & DFP & BFGS \\\hline
	是否准确收敛 & $\surd$ & $\surd$ & $\surd$ & $\times$ & $\times$ \\\hline
	迭代次数 & 12 & 12 & 1001 & - & 121 \\\hline
	函数调用次数 & 38 & 38 & 449556 & - & 4713 \\\hline
	CPU时间 & 0.0156 & 0 & 24.1250 & - & 0.5156\\\hline
\end{tabular}

\vspace{20pt}
n=12,最大迭代次数1000,$\epsilon=10^{-10}$

\vspace{10pt}
\begin{tabular}{|l|c|c|c|c|c|}
	\hline
	方法 & 阻尼 & LM & SR1 & DFP & BFGS(取$\epsilon=10^{-100}$) \\\hline
	是否准确收敛 & $\surd$ & $\surd$ & $\times$ & $\times$ & $\surd$ \\\hline
	迭代次数 & 12 & 12 & - & - & 1001 \\\hline
	函数调用次数 & 38 & 38 & - & - & 308719 \\\hline
	CPU时间 & 0.0156 & 0.0156 & - & - & 20.6719\\\hline
\end{tabular}

\subsubsection{Strong Wolfe准则}

n=6,最大迭代次数1000,$\epsilon=10^{-10}$

\vspace{10pt}
\begin{tabular}{|l|c|c|c|c|c|}
	\hline
	方法 & 阻尼 & LM & SR1 & DFP & BFGS \\\hline
	是否准确收敛 & $\surd$ & $\surd$ & $\surd$ & $\times$ & $\surd$ \\\hline
	迭代次数 & 11 & 11 & 36 & - & 46 \\\hline
	函数调用次数 & 35 & 35 & 102109 & - & 3329 \\\hline
	CPU时间 & 0 & 0 & 4.2656 & - & 0.1563\\\hline
\end{tabular}

\vspace{20pt}
n=9,最大迭代次数1000,$\epsilon=10^{-10}$

\vspace{10pt}
\begin{tabular}{|l|c|c|c|c|c|}
	\hline
	方法 & 阻尼 & LM & SR1($\epsilon=10^{-100}$) & DFP & BFGS($\epsilon=10^{-100}$) \\\hline
	是否准确收敛 & $\surd$ & $\surd$ & $\surd$ & $\times$ & $\surd$ \\\hline
	迭代次数 & 12 & 12 & 44 & - & 115 \\\hline
	函数调用次数 & 38 & 38 & 202042 & - & 7182 \\\hline
	CPU时间 & 0.0156 & 0.0156 & 11.0469 & - & 0.5781\\\hline
\end{tabular}

\vspace{20pt}
n=12,最大迭代次数1000,$\epsilon=10^{-10}$

\vspace{10pt}
\begin{tabular}{|l|c|c|c|c|c|}
	\hline
	方法 & 阻尼 & LM & SR1($\epsilon=10^{-100}$) & DFP & BFGS($\epsilon=10^{-100}$) \\\hline
	是否准确收敛 & $\surd$ & $\surd$ & $\surd$ & $\times$ & $\surd$ \\\hline
	迭代次数 & 12 & 12 & 74 & - & 1001 \\\hline
	函数调用次数 & 38 & 38 & 17323 & - & 624231 \\\hline
	CPU时间 & 0.0313 & 0.0313 & 21.3594 & - & 43.5938\\\hline
\end{tabular}

\subsubsection{Goldstein准则}

n=6,最大迭代次数1000,$\epsilon=10^{-10}$

\vspace{10pt}
\begin{tabular}{|l|c|c|c|c|c|}
	\hline
	方法 & 阻尼 & LM & SR1 & DFP & BFGS \\\hline
	是否准确收敛 & $\surd$ & $\surd$ & $\surd$ & $\times$ & $\surd$ \\\hline
	迭代次数 & 11 & 11 & 24 & - & 40 \\\hline
	函数调用次数 & 24 & 24 & 50 & - & 82 \\\hline
	CPU时间 & 0 & 0 & 0.625 & - & 0.9063\\\hline
\end{tabular}

\vspace{20pt}
n=9,最大迭代次数1000,$\epsilon=10^{-10}$

\vspace{10pt}
\begin{tabular}{|l|c|c|c|c|c|}
	\hline
	方法 & 阻尼 & LM & SR1($\epsilon=10^{-100}$) & DFP & BFGS($\epsilon=10^{-100}$) \\\hline
	是否准确收敛 & $\surd$ & $\surd$ & $\surd$ & $\times$ & $\surd$ \\\hline
	迭代次数 & 12 & 12 & 1001 & - & 154 \\\hline
	函数调用次数 & 26 & 26 & 2002 & - & 310 \\\hline
	CPU时间 & 0.0156 & 0 & 71.6563 & - & 4.9531\\\hline
\end{tabular}

\vspace{20pt}
n=12,最大迭代次数1000,$\epsilon=10^{-10}$

\vspace{10pt}
\begin{tabular}{|l|c|c|c|c|c|}
	\hline
	方法 & 阻尼 & LM & SR1 & DFP & BFGS \\\hline
	是否准确收敛 & $\surd$ & $\surd$ & $\times$ & $\times$ & $\times$ \\\hline
	迭代次数 & 12 & 12 & 60 & - & 635 \\\hline
	函数调用次数 & 26 & 26 & 122 & - & 1252 \\\hline
	CPU时间 & 0.0156 & 0.0156 & 15.8438 & - & 377.5156\\\hline
\end{tabular}

\section{结论}

首先,DFP公式基本可以被淘汰。(不排除我代码问题的可能性,未来得及复验)

LM(修正牛顿方法)在实践中和拟牛顿方法效果基本相同,因为它主要是解决Hessian矩阵的非正定问题,而这个问题中的Hessian矩阵基本上正定,所以不会收到影响。

拟牛顿方法所需迭代次数一般比阻尼牛顿法要高,而且调用函数次数也远远高于阻尼牛顿与修正牛顿方法。在这个问题中,Goldstein准则是一个比较好的选择步长的条件

最速下降法(梯度下降)收敛速度极慢(代码中有验证,报告未体现),所以通过将梯度下降和拟牛顿法结合的修正牛顿方法并不会取得很好的效果。

\section{后记}
\subsection{版本修正说明}
关于Wolfe准则的参数选择,增加了参考文献。对于拟牛顿方法初始迭代矩阵的选择,引用了参考文献加以说明。

放弃了每次函数调用都计算Hessian矩阵的做法,提高了算法的速度。并更新了数据。

增加了关于回溯法寻找步长的步长初始值的实验和分析

在文末添加了致谢和一堆废话

\subsection{需要改进的实验内容}

精确线搜索的步长是一个很难的问题。有些同学采用书中\cite{高立2014数}所述的0.618法,因为没有证明函数的凸性与单峰,所以只能作为“近似”的方法。插值法也不能求得精确的函数。所以在我的实验中只能采取一些调用库函数的方法。

从文献\cite{Nocedal2006NO}中发现了一种使用插值法求得满足Wolfe条件的搜索步长的方法。仍然是一种非精确搜索的方法。这个方法没有付诸实践。

在网上搜集资料的时候发现了一些近些年新的拟牛顿方法\cite{WEI20061156},没有进行尝试

虽然题目中没有要求,但是没有对比线搜索与信赖域方法是本文的一大缺陷。

\subsection{实验过程}
这个项目是搞了很久的。基本上经历了MATLAB从入门到入门的过程。初期的时间主要用来思考“我写的程序究竟是不是正确的”
这个哲学问题。后来使用了几种不同的语言实现了同一个代码,并且验证了许多小的例子。

平时的工作不认真加上自己很菜,导致手算梯度转化成代码的时候遇到一些BUG。自己给自己挖了很大的一个坑,浪费了大把大把宝贵的时间在这里面。很大一部分时间用来查找参考文献,从中确实也学到了很多知识。(写完这个我甚至怀疑我是不是隔壁情报检索专业偷偷混进来的)
。doi上面有两篇看上去很不错的期刊文章,可惜中大图书馆没买,荐购了。

还有我查了好久好久,都没找到DFP公式应该在哪些优化实例得到较好的结果。这个我不知道应该写在结论里还是写在这里。您可能也明白了,这一部分是整篇报告我留给自己吐槽的地方。毕竟不是严谨的学术论文,有什么话不妨就说一说呗。反正阁下也不用看这部分,因为从这句开始就全都是废话了。

我是在抑郁发作期写完这个东西的。真的很痛苦。总觉得自己是个智障。事实证明自己就是个智障。

之前尝试用TensorFlow之类的现代工具去验证。比如尝试了新时代的AdamOptimizer。但是并不能精确地收敛到我们想要的值。在速度和精度之间取舍始终是数值优化领域的难题。

看了那许多论文,哇塞,数值优化这个学科是真强。做做实验推推公式才能感受到世界的美好。

这十多天我一直在思考一个哲学问题。我的程序跑出来一个结果,我怎么证明他是对的?毕竟最后n=12的时候最优值精度已经到了$10^{-10}$这个数量级。这玩意跟0有什么区别?我觉得它只配当个$\epsilon$

但是后来事实证明我错了,原来我的$\epsilon$是可以取到$10^{-100}$的。我感觉这个就是个停机问题,无解,就是一种煎熬。我现在能找到文献和参考值。那么将来呢?我怎么知道我的程序就是正确的?

虽然做了很久,但是仍然有许多许多没有实现。有人说我太追求极端和完美。我也不知道这对不对。总觉得吧,能力太差,烧多少时间都白搭。

完蛋。写到这里之后我又陷入了哲学般的思考。这对一个抑郁患者不是什么好事。没办法,我本来不想写这个的,但是谁叫你们把DDL推迟了一天呢?

总之,经过这次实验,我深深感受到了自身能力的匮乏。请多多指教。

\subsection{致谢}
感谢钱宇猷\ 胡君慈\ 刘明昊\ 庞泽雄等同学一起交流试验结果,使得各自对实验结果心里有个数。

感谢中山大学图书馆订购的文献。

感谢老师布置这次作业让我重新认识自我思考人生。

\addcontentsline{toc}{part}{参考文献}
\bibliographystyle{unsrt}
\bibliography{notebio}
\end{document}